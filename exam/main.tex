\documentclass{ksexam}

\pointname{\,$\pi$}

\begin{document}

\hfset{数学测试试卷}
{学号:\underline{\hspace*{8em}}}
{姓名:\underline{\hspace*{8em}}}
{\thepage{}\ /\ \numpages{}}
\head{数学测试试卷}

\begin{center}
    \hqword{题号}
    \hpword{分值}
    \htword{总分}
    \hsword{得分}
    \cellwidth{0.8em}
    \gradetable[h][questions]
\end{center}

\begin{questions}

    \sect{选择题}

    \question[10] 下列哪个分数是最简分数?(\hspace*{2em})

    \begin{oneparchoices}
        \choice $\dfrac{2}{4}$
        \choice $\dfrac{3}{9}$
        \CorrectChoice $\dfrac{5}{6}$
        \choice $\dfrac{8}{12}$
    \end{oneparchoices}

    \question[10] 下列哪个算式是正确的?(\hspace*{2em})

    \begin{choices}
        \choice $3 \times (4 + 5) = 27$
        \CorrectChoice $3 \times (4 + 5) = 3 \times 4 + 3 \times 5$
        \choice $3 \times (4 + 5) = 4 \times (3 + 5)$
        \choice $3 \times (4 + 5) = (3 + 4) \times (3 + 5)$
    \end{choices}

    \sect{填空题}

    \question[8] 把下列分数按从小到大的顺序排列,并用“$<$”号连接起来:$\frac{1}{2}$,$\frac{2}{3}$,$\frac{3}{4}$,$\frac{4}{5}$

    答:$\frac{1}{2} < \underline{\hspace{1cm}} < \underline{\hspace{1cm}} < \underline{\hspace{1cm}}$

    \question[8] 计算下列算式的值,并把结果写在括号里:$6 \times 7 = (\quad)$,$8 \div 4 = (\quad)$,$9 + 9 - 9 = (\quad)$

    答:$6 \times 7 = (\underline{\hspace{1cm}})$,$8 \div 4 = (\underline{\hspace{1cm}})$,$9 + 9 - 9 = (\underline{\hspace{1cm}})$

    \question[8] 把下列小数化成百分数,并在等号后面填上答案:$0.25 = \underline{\hspace{1cm}} \%$,$0.75 = \underline{\hspace{1cm}} \%$,$1.00 = \underline{\hspace{1cm}} \%$

    答:$0.25 = \underline{\hspace{1cm}} \%$,$0.75 = \underline{\hspace{1cm}} \%$,$1.00 = \underline{\hspace{1cm}} \%$

    \question[14] 小明有20元钱,他买了一本书和两支笔,一共花了16元钱。请问他还剩下多少钱?

    答:小明还剩下\underline{\hspace{2cm}}元钱。

    \question[14] 小华有15个苹果,他把其中的一半分给了小丽,然后又从小丽那里得到了两个苹果。请问小华最后有多少个苹果?

    答:小华最后有\underline{\hspace{2cm}}个苹果。

    \question[14] 小红有一条长为60厘米的绳子,她把它剪成了三段,每段的长度都是整数厘米,并且第一段的长度是第二段的长度的两倍,第二段的长度是第三段的长度的三倍。请问三段绳子的长度各是多少厘米?

    答:三段绳子的长度分别是\underline{\hspace{2cm}}厘米,\underline{\hspace{2cm}}厘米和\underline{\hspace{2cm}}厘米。

    \question[14] 小刚有一个正方形的纸板,每条边长为10厘米,他在纸板的四个角上各剪掉一个边长为2厘米的小正方形,然后把剩下的部分折成一个无盖的盒子。请问这个盒子的体积是多少立方厘米?

    答:这个盒子的体积是\underline{\hspace{2cm}}立方厘米。

\end{questions}

\end{document}
